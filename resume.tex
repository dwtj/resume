%%%%%%%%%%%%%%%%%%%%%%%%%%%%%%%%%%%%%%%%%
% Medium Length Graduate Curriculum Vitae
% LaTeX Template
% Version 1.1 (9/12/12)
%
% This template has been downloaded from:
% http://www.LaTeXTemplates.com
%
% Original author:
% Rensselaer Polytechnic Institute (http://www.rpi.edu/dept/arc/training/latex/resumes/)
%
% Important note:
% This template requires the res.cls file to be in the same directory as the
% .tex file. The res.cls file provides the resume style used for structuring the
% document.
%
%%%%%%%%%%%%%%%%%%%%%%%%%%%%%%%%%%%%%%%%%

%------------------------------------------------------------------------------
%	PACKAGES AND OTHER DOCUMENT CONFIGURATIONS
%------------------------------------------------------------------------------

\documentclass[margin, 10pt]{res} % Use the res.cls style, the font size can be
                                  % changed to 11pt or 12pt here

%\usepackage{helvet} % Default font is the helvetica postscript font
%\usepackage{newcent} % To change the default font to the new century schoolbook
                      % postscript font uncomment this line and comment the one
                      % above.
%\usepackage[sfdefault]{roboto}
\usepackage{times}

\usepackage{hyperref}

\setlength{\textwidth}{5.3in} % Text width of the document

\begin{document}

%------------------------------------------------------------------------------
%	NAME AND ADDRESS SECTION
%------------------------------------------------------------------------------

\moveleft.5\hoffset\centerline{\large\bf David Johnston} % Your name at the top
\moveleft.5\hoffset\centerline{Software Engineer} % Your name at the top

 % Horizontal line after name; adjust line thickness by changing the '1pt'
\moveleft\hoffset\vbox{\hrule width\resumewidth height 1pt}\smallskip

\moveleft.5\hoffset\centerline{\href{mailto:dwtj@iastate.edu}{dwtj@iastate.edu}}
\moveleft.5\hoffset\centerline{\url{https://github.com/dwtj}}

%------------------------------------------------------------------------------

\begin{resume}

%------------------------------------------------------------------------------
%	EDUCATION SECTION
%------------------------------------------------------------------------------

\section{EDUCATION}

{\sl Bachelor of Engineering,} Major in Software Engineering (GPA 3.5) \\
Iowa State University, Ames, IA. (Expected December 2015)

{\sl Bachelor of Science,} Major in Mathematics + Computer Science (GPA 3.5) \\
Iowa State University, Ames, IA. (Expected December 2016)

{\sl Master of Science,} Computer Science \\
Iowa State University, Ames, IA. (Expected May 2017)


%------------------------------------------------------------------------------
%	COMPUTER SKILLS SECTION
%------------------------------------------------------------------------------

\section{COMPUTER \\ SKILLS}

{\sl Advanced Java:} compiler plugins (via annotation processors), reflection,
  metaprogramming, concurrency, instrumentation, program analysis (via IBM's
  WALA).

{\sl Misc:} Unix/Linux/BSD system administration; C/C++; formal verification and
  model checking (Spin); formal proof assistants (Coq); data analysis (R,
  Python, and MATLAB); Python network programming (Twisted).

%------------------------------------------------------------------------------
%	PROFESSIONAL EXPERIENCE SECTION
%------------------------------------------------------------------------------

\section{EXPERIENCE}

{\sl ISU Graduate Research Assistant} \hfill {\sl January 2015-Present} \\
  \href{http://design.cs.iastate.edu/}{Laboratory for Software Design}
  \begin{itemize} \itemsep -1pt % Reduce space between items
    \item Lead team developing @PaniniJ (\url{https://github.com/hridesh/panini}),
          a language embedded in Java for safe concurrent programming.
    \item Research techniques for novel ownership transfer check (both static
          and dynamic) for preventing certain race conditions in concurrent
          systems.
    \item Present recent recearch results from the SE and PL literature at
          lab meetings.
  \end{itemize}

{\sl ISU Graduate-Course Teaching Assistant}         \hfill {\sl Spring 2013}
  \begin{itemize} \itemsep -1pt
    \item HCI 575, {\sl Computational Perception}: Created computer vision
          assignments involving MATLAB, OpenCV, and PCL; guest lectured on
          programming with OpenCV.
  \end{itemize}

{\sl ISU Undergraduate-Course Teaching Assistant}    \hfill {\sl Spring 2012,
                                                     Fall 2013, Spring 2014}
  \begin{itemize} \itemsep -1pt
    \item CprE 185, {\sl Introductory C}: Co-lectured weekly recitations;
          guest-lectured on the topic of pointers.
    \item CprE 281, {\sl Digital Logic}: Co-lectured weekly recitations;
          guest-lectured on the topic of hardware-implemented finite state
          machines.
    \item ComS 319, {\sl Software Construction and User Interfaces}: Assisted
          students as they completed assignments and as they developed
          semester-long group projects.
  \end{itemize}


%------------------------------------------------------------------------------
%	COMMUNITY SERVICE SECTION
%------------------------------------------------------------------------------

\section{COMMUNITY \\ SERVICE}

{\sl ISU SE SAC Chair:} Led the ISU Software Engineering Student Advisory
  Council.\\
\\
{\sl ISU GPSS Senator:} Representative for the CS graduate program's in the ISU
  Graduate and Professional Student Senate.

%------------------------------------------------------------------------------
%	Journal Publications
%------------------------------------------------------------------------------

\section{PUBLICATIONS}

Schenck, C.; Sinapov, J.; Johnston, D.; Stoytchev, A., ``Which Object Fits Best?
Solving Matrix Completion Tasks with a Humanoid Robot," in {\sl Autonomous
Mental Development}, IEEE Transactions on , vol.6, no.3, pp.226-240, Sept. 2014

\end{resume}
\end{document}
